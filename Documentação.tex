
\documentclass[12pt]{article}
\usepackage[utf8]{inputenc}
\usepackage[brazil]{babel}
\usepackage{graphicx}
\usepackage{amsmath}
\usepackage{amssymb}
\usepackage{tikz}
\usepackage{hyperref}
\usetikzlibrary{positioning}
\usepackage{geometry}
\usepackage{newtxtext,newtxmath}
\geometry{a4paper, margin=2.5cm}

\title{Trabalho de Pilhas e Filas\\\large Estruturas de Dados e Análise de Algoritmos}
\author{Lucas Gabriel Rodrigues Valadares - RA: 92310851}
\date{\today}

\begin{document}

\maketitle
\tableofcontents
\newpage

\section{Introdução}

Este trabalho tem como objetivo desenvolver duas aplicações utilizando as estruturas de dados \textbf{pilhas} e \textbf{filas}, explorando sua aplicabilidade em situações práticas. Para isso, foram escolhidos dois cenários bastante presentes no cotidiano: um navegador de internet e um sistema de gerenciamento de senhas para atendimento hospitalar.

No primeiro caso, foi implementado um navegador que simula o funcionamento dos botões \textit{Voltar} e \textit{Avançar}, utilizando duas pilhas para gerenciar o histórico de navegação de forma eficiente. No segundo, foi desenvolvido um sistema de filas para um hospital, capaz de gerar senhas sequenciais, organizar a ordem de atendimento e manter um histórico das senhas chamadas.

A escolha desses dois cenários permitiu não apenas explorar as funcionalidades básicas das estruturas estudadas, mas também refletir sobre sua utilidade na solução de problemas reais, contribuindo para a consolidação dos conceitos de estrutura de dados e programação orientada a objetos.
\\

O código-fonte completo está disponível em: \\\texttt{\url{https://github.com/Lucasvalada12/Atividade-Fila-Pilha}}

\section{Implementação}

\subsection{Estruturas Utilizadas}
\begin{itemize}
    \item \textbf{Pilha (Stack)}: utilizada para armazenar o histórico de navegação (páginas anteriores e próximas).
    \item \textbf{Fila (Queue)}: utilizada para gerenciar a fila de atendimento dos pacientes no hospital.
\end{itemize}

\subsection{Sistema de Navegador}

O navegador simula o comportamento dos botões \textbf{voltar} e \textbf{avançar}, permitindo navegar entre páginas visitadas utilizando duas pilhas.

\subsubsection*{Diagrama de Funcionamento}

\begin{center}
\begin{tikzpicture}[
    node distance=4cm,
    auto,
    >=latex,
    every node/.style={font=\small}
]
    \node (voltar) [draw, rectangle] {Pilha Voltar};
    \node (pagAtual) [draw, rectangle, right=of voltar] {Página Atual};
    \node (avancar) [draw, rectangle, right=of pagAtual] {Pilha Avançar};
    
    \draw[->] (pagAtual) -- node[above] {Botão Voltar} (voltar);
    \draw[->] (voltar) -- node[below] {Mover para Avançar} (pagAtual);
    \draw[->] (pagAtual) -- node[above] {Botão Avançar} (avancar);
    \draw[->] (avancar) -- node[below] {Mover para Voltar} (pagAtual);
\end{tikzpicture}
\end{center}

\subsection{Sistema de Senhas}
O sistema permite gerar senhas sequenciais, chamar o próximo paciente da fila e manter o histórico de senhas chamadas.

\section{Casos de Teste}

\subsection*{Navegador}
\begin{itemize}
    \item Acessar páginas: google.com, inkarnate.com, outlook.live.com
    \item Voltar duas vezes e avançar uma vez
\end{itemize}

\subsection*{Sistema de Senhas}
\begin{itemize}
    \item Gerar três senhas
    \item Chamar duas senhas
    \item Exibir o histórico
\end{itemize}

\section{Conclusão}

A implementação dos dois sistemas propostos neste trabalho permitiu aplicar de forma prática os conceitos de pilhas e filas, reforçando seu papel fundamental na resolução de problemas de controle e organização de dados. Além disso, a utilização de programação orientada a objetos proporcionou uma estrutura modular e de fácil manutenção, favorecendo a clareza e a escalabilidade do código.

Entre os principais desafios enfrentados, destaca-se a manipulação simultânea das duas pilhas no sistema do navegador, especialmente na lógica de alternância entre os botões de avançar e voltar. Essa etapa exigiu um cuidado especial na definição das operações de empilhar e desempilhar para garantir o comportamento correto da aplicação.

Por fim, a experiência adquirida neste projeto foi essencial para aprofundar o entendimento teórico e prático de estruturas de dados, além de reforçar a importância de projetar soluções robustas e bem estruturadas para problemas reais.



\section{Bibliografia}
\begin{itemize}
    \item Cormen, T. H., Leiserson, C. E., Rivest, R. L., \& Stein, C. (2009). \textit{Algoritmos: teoria e prática}.
    \item Slide sobre Listas Lineares.
     \item Documentação oficial do Java: \url{https://docs.oracle.com/javase/8/docs/}
    \item Documentação de LaTeX: \url{https://www.learnlatex.org}
\end{itemize}

\end{document}
